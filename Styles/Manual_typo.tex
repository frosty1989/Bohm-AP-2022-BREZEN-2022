\chapter{Citování a~formální náležitosti AP\label{ch:Manual_typo}}





Tato práce je napsána jako šablona pro absolventskou práci. Všimněte si zejména stylu dokumentu: zarovnání textu na obou stranách, popisy obrázků, číslování rovnic, číslování a~formátování kapitol, seznam literatury, odkazy na vztahy, obrázky atd.

V~toté příloze jsou shrnuta pravidla citování a~stručně uvedeny formální požadavky na práci spolu s~několika typografickými zásadami, které publikační systém \LaTeX\/ dodržuje sám. Více naleznete v~pří\-sluš\-ných normách. 



\section{Citování a~plagiátorství \label{se:Manual_cite}}

Využití cizích myšlenek upravuje citační etika a~normy! Nejprve ale uveďme základní pojmy, jako je citát, parafráze, citace a~citační odkaz~\cite{AbsolventThesis:FRANCIREK_AP}.
\begin{itemize}
    \item \textbf{Citát} je citovaná myšlenka nebo výrok jiného autora. Jde o~přímý a~doslovný přepis přebíraného obsahu.
    \item \textbf{Parafráze} je nepřímou variantou citátu. To znamená, že myšlenky jiného autora neopisujete doslovně, ale přepisujete je vlastními slovy.
    \item \textbf{Citace} neboli bibliografický záznam (zde je to kapitola Literatura) uvádí údaje o~citovaném zdroji (autory, název, místo vydání, rok vydání atd.).
    \item \textbf{Citační odkazy} spojují citáty či parafráze s~citacemi (zde je tvoříte příkazem \verb"\cite{}").
    \item \textbf{Plagiát} je dílo, ve kterém autor nepřizná použití cizích zdrojů. Jde o~neetické dílo odpovídající nezákonému obohacování se v~běžném životě.
    \item \textbf{Kompilát} je neužitečný přepis (byť i~správně citovaný) obsahu jiných děl po\-strá\-da\-jí\-cí váš přínos.
\end{itemize}

Tak, jak si můžete doslovně přečíst v~\cite{AbsolventThesis:FRANCIREK_AP}\?: \uv{Každá odborná práce stojí na dosavadním poznání těch, kteří se problematice věnovali před vámi, proto je využití jejich poznatků logické, žádoucí a~očekávané. Užitím citátů ukazujete, že znáte práce renomovaných odborníků, že na jejich práci navazujete a~že uznáváte jejich význam pro své vlastní poznání. Čtenáři umožňujete každý citát dohledat a~dozvědět se o~něm více.} Více si přečtěte v~\cite[kapitola~2.3.4]{AbsolventThesis:FRANCIREK_AP}.


\subsection{Několik rad k~citování}

\begin{itemize}
    \item \textbf{Je naprosto nepřípustné kopírovat cizí texty!} Samozřejmě, Ohmův zákon ne\-mů\-že\-te napsat jinak, než zní. Ale nesmíte spolu s~ním zkopírovat z~nějaké knihy či webové stránky celý odstavec nebo dokonce celou stránku (zákon včetně průvodních textů).
    \item Pokud používáte výsledky druhých autorů ve větším rozsahu, \textbf{je nepřípustné tupě kopírovat jejich texty.} Ty musíte parafrázovat a~opatřit je citačním odkazem. 
    \item Pokud je to bezpodmínečně nutné a~nějaká věta parafrázovat nelze (například proto, že by se ztratil nebo znepřesnil její význam), pak to lze vyřešit eticky následujícím způsobem. \textcolor{blue}{Jak autor/autoři ve své publikaci (konkrétní citační odkaz) uvádí\?: \uv{Tady bude zkopírovaná věta, maximálně krátký odstavec.}} Takto samozřejmě nelze zkopírovat každý druhý odstavec do Vaší AP.
\end{itemize}


\subsection{Relevantnost a~kvalita citací}

Obecně lze seřadit citační zdroje takto\?:
\begin{enumerate}
    \item \textbf{knihy}, zejména ty, které prošly kvalitním recenzním řízením (obvykle před rokem 2000) -- ne\-vý\-ho\-dou je samozřejmě jejich stáří, ale pokud jde o~základní učebnice, pak jsou to velmi kvalitní zdroje
    \item \textbf{články v~renomovaných časopisech} (zejména zahraničních) -- zde je výhoda v~jejich aktuálnosti a~vysoké odbornosti
    \item \textbf{články v~odborných časopisech}
    \item \textbf{články z~vědeckých konferencí}
    \item \textbf{diplomové práce}
    \item \textbf{absolventské/bakalářské práce}
    \item \textbf{odborná internetová fóra} -- pozor na kompetence autora
    \item \textbf{Wikipedie} -- tu necitujte, je velmi dobrá na první hledání, ale posléze je třeba najít informaci ve výše zmíněných zdrojích.
\end{enumerate}



\section{Shrnutí formálních požadavků}

\begin{enumerate}
    \item[1)] \emph{Srovnání textu}:
    \begin{enumerate}
        \item[a)] zarovnávejte text na obou stranách, vypadá to lépe;
        \item[b)] číslujte stránky;
        \item[c)] nevynechávejte volné půlstránky (čtvrtstránky), například pod obrázky (je to naprosto nepřípustné);
        \item[d)]  používejte odstavce v~rozumné míře (odstavec by měl mít více jak jednu větu, kapitola by měla mít více jak jeden odstavec atd.).
    \end{enumerate}
    \item[2)] \emph{Kapitoly}:
    \begin{enumerate}
        \item[a)] rozdělte práce přehledně do jednotlivých kapitol a~podkapitol, typicky: Úvod, pak kapitoly popisující problém, způsob Vašeho řešení, výsledky a~nakonec Závěr;
        \item[b)] číslujte kapitoly a~podkapitoly;
        \item[c)] použijte rozumné vertikální mezery kolem nadpisů kapitol.
    \end{enumerate}
    \item[3)] \emph{Rovnice a~matematika v~textu}:
    \begin{enumerate}
        \item[a)] rovnejte rovnice na střed, nebo na nějaký, stále stejný, tabulátor;
        \item[b)] číslujte rovnice, na které se odkazujete; číslování zarovnávejte zcela vpravo;
        \item[c)] matematika v~textu by měla být stejným stylem jako v~rovnicích (většinou kurzívou viz níže);
        \item[d)] fyzikální veličiny, například napětí \quantity{$u$}{V};
        \item[e)] fyzikální veličiny s~hodnotou, například $u=5\.4\,$V (všimněte si, jak napsat desetinnou čárku a~mezery mezi číslem a~jednotkou).
    \end{enumerate}
    \item[4)] \emph{Obrázky a~grafy}:
    \begin{enumerate}
        \item[a)] číslujte a~vhodně popisujte obrázky (například místo popisu frekvenční charakteristika napište hlavně čeho je to frekvenční charakteristika, jakého přenosu atd.);
        \item[b)] popisy obrázků zarovnávejte na střed nebo na nějaký, stále stejný, tabulátor;
        \item[c)] v~grafech popisujte česky (ne anglicky v~české práci) osy, přidejte legendy, vhodně zvolte tloušťku čar, nemíchejte dva neporovnatelné grafy (jablka, hruš\-ky), apod;
        \item[d)] nepoužívejte nekvalitně nascanované obrázky, nakreslete si je sami.
    \end{enumerate}
    \item[5)] \emph{Tabulky}:
    \begin{enumerate}
        \item[a)] podobně jako obrázky; popis většinou nad tabulku.
    \end{enumerate}
    \item[6)] \emph{Literatura}:
    \begin{enumerate}
        \item[a)] používejte odkazy na literaturu, nemusíte tak vysvětlovat spoustu pojmů;
        \item[b)] zdroj, na který se v~textu neodkazujete, nemá v~seznamu literatura co dělat;
        \item[c)] nadpis Literatura bývá obvykle bez čísla.
    \end{enumerate}
\end{enumerate}


Důraz klaďte na přehlednost, přesnost, stručnost a~ucelenost. Uvědomte si, že práci píšete i~pro někoho, kdo o~Vašem problému nemusí mít zdání! Popište tedy důkladně Váš problém. Popište, jak budete problém řešit a~jaké očekáváte řešení. Nesnažte se zakrýt tyto věci množstvím vzorečků a~okopírovaného textu z~internetu apod. Nejdůležitější kapitoly jsou úvod, kde je popsán problém a~závěr, kde shrnete Vaše výsledky. Porovnáte předpoklady se skutečností, vyhodnotíte všechny body zadání případně zdůvodníte proč jste některé body nevypracovali a~navrhnete možné řešení.

Práce nesmí na první pohled vypadat odpudivě, ale naopak poutavě. Jinak by si ji ni\-kdo nikdy nepřečetl a~i~vedoucí a~oponent ji bude číst s~předsudkem, že je to špatná práce! Práce by měla být i~slohově pěkná. Dejte ji přečíst někomu dalšímu a~pokud on ji dobře porozumí, je to dobrý základ proto, aby byla dobře ohodnocena. Také není od věci, nechat si práci jazykově a~gramaticky zkontrolovat nějakým odborníkem.



\section{Několik typografických zásad}

Čerpáno z~materiálů prof.~RNDr.~Petra Kulhánka,~CSc.\?: pravidla psaní výrazů. Sazba je určena mezinárodními normami (nemůžete psát texty, jak se Vám zlíbí)\?:
\begin{itemize}
    \item ISO\?: http://www.iso.ch/
    \item ČSN ISO\?: http://shop.normy.biz/
\end{itemize}

\noindent Sazba matematických výrazů\?:
\begin{itemize}
    \item normální (stojatá) sazba\?: pro čísla, funkce ($\sin,\, \cos$ apod.), číselné indexy, zkratky ($\max,\, \min$ apod.), jednotky (kg, N apod.), transpozici matice
    \item {\em italika (kurzíva)}\?: pro veličiny, proměnné, konstanty, indexy jako proměnné apod.
    \item \textbf{tučná sazba}\?: vektory, matice
    \item řecká písmena\?: proměnné italikou, suma apod. vyjádřená velkým řeckým Sigma normálním (stojatým) písmem
\end{itemize}

\noindent Mezery\?:
\begin{itemize}
    \item mezi číslem a~jednotkou (jednotky se nesmějí na konci řádku odělit od čísel)
    \item mezi jednotkami (1/4)
    \item před a~za rovnítkem
    \item argument funkce neuzavřený do závorky $(\sin 2x)$
    \item velikosti mezer v~matematice\?:
    \begin{itemize}
        \item \verb"\," 1/4
        \item \verb"\>" 1/2
        \item \verb"\;" 3/4
        \item \verb"\"  4/4 (celá)
        \item \verb"\quad"
        \item \verb"\qquad"
    \end{itemize}
    \item velikosti mezer v~textu\?:
    \begin{itemize}
        \item \verb"\," 1/4
        \item \verb"\/" italické vyrovnání
        \item \verb"\" 4/4 (celá)
        \item \verb"~" pevná
    \end{itemize}
\end{itemize}



\textcolor{blue}{\em Tento text se nachází v~souboru \texttt{\cestaStyles Manual\_typo.tex}. Pro odstranění této přílohy zakomentujte v~souboru \texttt{Diplomka.tex} řádek~171, respektive v~souboru \texttt{MP.tex} řádek~181, respektive v~souboru \texttt{SOC.tex} řádek~181\?: \texttt{$\backslash$input$\{$\cestaStyles Manual\_Typo.tex$\}$}.\/} 
