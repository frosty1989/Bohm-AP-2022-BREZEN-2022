\chapter{Introduction \label{ch:uvod}}




Laser shock peening (LSP) is a surface treatment process used to prolong the fatigue life of metallic components. LSP is successfully used in industrial applications, and there is a tremendous commercial interest in applying LSP. The layout of most modern LSP stations comprises a laser source and an industrial robotic arm holding the sample to be processed. One such LSP station is located at the HiLASE centre, Czech Republic. LSP is suitable for treating parts of complex geometry, such as forging dies and cutters. However, this advantage comes with a new challenge: to program the robotic arm in such a way as to ensure an optimal LSP process on parts with complex geometry. These programs can be generated by using Computer-Aided Manufacturing (CAM) software. A critical feature of every CAM software is the post processor. A post processor defines the vendor-specific rules a robotic arm program must follow.  In other words, post processors serve as tools to convert the simulation to vendor-specific robotic arm programs \cite{ding_ye_2006}.

The potential of high-energy pulsed laser systems to create plastic deformation was first discovered by White in 1963 in the USA \cite{white_1963}. The confined ablation mode in air was first recognised by Neuman in 1964 \cite{neuman_1964}. During the 1970s, research into the possible applications of LSP continued. During the next decade, in the 1980s, a team centred around Remy Fabbro at the Ecole Polytechnique made significant theoretical and experimental contributions to LSP \cite{fabbro_fournier_ballard_devaux_virmont_1990}. At the same time, researchers at the Battelle Memorial Institute in Columbus, Ohio, developed a prototype pulsed laser to demonstrate the industrial viability of LSP. The 1990s marked the decade when LSP saw its first commercial applications. Notably, General Electric Aviation used LSP to solve Foreign Object Damage (FOD) on fan blades of turbofan jet engines \cite{airforce}. The 2000s and 2010s saw the rise of companies performing LSP commercially \cite{sano}.

The goals of this study can be summarised into the following points:
\begin{enumerate}

    \item Create a simulation of a LSP process in CAM software
    \item Develop a post processor for the LSP process 
    \item Generate a program  for the robotic arm controller from simulation
    \item Test simulation of LSP process on physical robotic arm

    
\end{enumerate}

This study is divided into the following chapters: \hyperref[chap:basics]{Chapter 2} contains the basics of industrial robotics. \hyperref[chap:peening]{Chapter 3} acquaints the reader with the LSP process.  It describes the research topic, related work, and the experimental setup.  
\hyperref[chap:design]{Chapter 4} deals with the CAM program used in this study, RoboDK. The actual implementation of the post processor in the programming language Python is the focus of \hyperref[chap:implementation]{Chapter 5}. In \hyperref[chap:testing]{Chapter 6}, the post processor for the LSP process is tested on an actual industrial robotic arm setup. Finally, the closing \hyperref[chap:discussion]{Chapter 7} contains the conclusions and future work. 

The HiLASE research centre is an infrastructure focused on laser research and development, located in Dolní Břežany, Czech Republic. HiLASE's Industrial laser applications program (ILA) focuses on LSP, laser induced damage threshold (LIDT), and laser micromachining technologies. ILA is equipped with several experimental stations, including an LSP station. The author of this study is employed as a control system engineer in the laser shock peening group of the ILA program. In addition, the author is responsible for carrying out LSP experiments for industrial partners and universities. The author's other tasks include programming the LSP station's robotic arm, integrating different laser sources with the  LSP station's robotic arm and constructing and putting into operation subsystems of the LSP station.




